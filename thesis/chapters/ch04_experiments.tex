\chapter{Experimental Results}
\label{ch:experiments}

% This chapter should describe your experimental set up and evaluation. It should also produce and describe the results of your study. Possible section titles are given below.

\section{Experimental Design}

\subsection{Program Tests}



\subsection{User Tests}

\subsubsection{Institutional Review Board}

The form of user tests are surveys. To conduct the user servey, the designed survey
have to be approved by Allegheny College Institutional Review Board (IRB). Institutional
Review Board is an organization that reviews and supervises experiments to make
sure the human rights of participants do not get hurt \cite{irb}. Before the application
starts, experimentors have to finish a Collaborative Institutional Training Initiative
(CITI Program) \cite{citi}. The program required by the Allegheny College Institutional
Review Board is Social & Behavioral Research. To get approved by the IRB, a proposal
first have to be submitted. The proposal contains review questions to help experiment
designer and organizers to identify the type of experiment because certain types
of experiment can get exemption or expedition. The questions are, for example,
does the experiment involves human, does the experiment involves individuals under
the age of 18. This experiment required exempted review. Other than the review
type, a brief introduction, procedure, and walk-through should be included.

\subsubsection{Description and Rationale of Project}

The description and rationale of project included in the Institutional Review
Board proposal are as follows:

This develops a novel software system to recommend songs to users. The mainstream
music service providers utilize user browsing and listening history along with
comparison of the user’s playlist to that of other users with similar habits to
recommend music to users.  In this project, instead of standard data, our system
uses machine learning to identify the user's current emotion by asking a user to
enter a text that contains some indication of their current emotions and feelings.
The text entered by the user can be their diary, a transcribed conversation, or
a script from a show that their current emotional state relates to. Using this
text, our system then learns about the users’ current emotional state and uses
that information to predict desired musical themes and songs to the user. The
purpose of the user study is to verify the efficiency of this software system,
where the participants are asked to provide a short body of text, and following
the review of the recommended song list they are invited to provide feedback on
the outcome they received.

\subsubsection{Methods and Procedures}

The methods and procedures of the experiment included in the Institutional Review
Board proposal are as follows:

The participation for this study will be advertised on My Allegheny, and via
departmental announcements (posters, club emails, Slack channel). The individuals
who agree to participate in the study will be emailed a link to the website on
which our system operates. Once on the landing page of the website, the participants
will be provided information about the experiment and they will be informed that
their consent is received if they proceed with the next steps of the study.  In
the next steps, the participants will create an account in our system and then
log in to their unique account. The walk-through section contains detailed sequence
of steps that the participants will be involved in and the pages on our website
that they will be directed to.

The website will be hosted on a private cloud-based server (Amazon Cloud Server, AWS).
The IP addresses and other information that could associate with the participants
will be deleted before analyzing the data. The data will be kept for four years
for possible further study, and it will be deleted after four years. AWS’s data
centers and network architecture is designed to protect information, identities
and applications stored in their cloud by following core security and compliance
requirements. The information required for the registration will be the username
and password. After logging in, the participants will be asked to follow the
instructions on the website and provide a body of text that reflects their emotions
on three different days. The following instructions will be provided:

\begin{displayquote}
  Please choose three days of this week, and for each day:

  \textbf{Option 1:} Provide more than a paragraph of text in any form (e.g. diary)
  that best describes your emotions or experiences on that day.

  \textbf{Option 2:} Provide more than a paragraph of a conversational or summarization
  text (e.g. TV scripts or TV scene summarization) that describes an emotional scene.

  Please provide a descriptive title to each body of text that you enter.
\end{displayquote}

After receiving the bodies of text from the user, our system will automatically
provide a list of recommended songs. After getting the recommended music playlist,
participants will be invited to participate in the survey by answering the following
three questions:

\begin{displayquote}
  \begin{enumerate}
    \item Describe the body of text that you provided. What emotions did you want
    to express in this text?
    \item Please choose the best description for the recommended playlist.
    \begin{enumerate}
      \item This playlist has nothing to do with my text
      \item I can see a rare relation to my text
      \item This playlist is acceptable, but I do not have a strong emotional
      connection to it
      \item I have a fair amount of an emotional connection to this playlist
      \item I have a strong emotional connection to this playlist
    \end{enumerate}
    \item If you feel an emotional connection to this playlist, why do you think
    the music is related to your provided text? If you do not feel a connection
    to this playlist, what do you think the playlist is missing?
  \end{enumerate}
\end{displayquote}

The goal of the questions in the survey above is to assess the efficiency of the
algorithm and the participants’ interpretation of their feelings. The answers will
be collected on the same website of our system and the answers will be connected
to specific user accounts. The user account is the only connection between the
answers and the provided bodies of text by the user. Again, all the user information,
including IP address, browser information, usernames, and passwords will be deleted
before the analysis of the collected data.

After the survey, the data will be analyzed using the responses in the user satisfaction
from question 2. Responses in the range from 1 to 3 will be marked as “need for
further analysis”. Question 1 and question 3 will be used to interpret the unsatisfied
playlist. The summary of the analyzed results will be reported in a senior thesis
document that will be stored on the cloud-based version-control system, called GitHub.
All participants who indicate an interest in receiving the results of the study
will be emailed a link to the GitHub page containing the senior thesis document.

The potential risks of this study is that the participants may spend a long time
writing texts. They  may also have to think about sad memories. If the participants
do not wish to proceed, they may quit any time during the process by pushing the
\emph{quit} button on the page.

\subsubsection{Walk-Through}

Participants will receive a link to begin their participation in this user study.
When participants click on the link, they will land on the “Information” page.

\subparagraph{Information Page}

After entering the website, users would see an information page that contains the
following message:

\begin{displayquote}
  \textbf{Project introduction:}

  This project develops a novel software system to recommend songs to users. The
  mainstream music service providers utilize user browsing and listening history
  along with comparison of the user’s playlist to that of other users with similar
  habits to recommend music to users. In this project, instead of standard data,
  our system uses machine learning to identify the user's current emotion by asking
  a user to enter a text that contains some indication of their current emotions
  and feelings. The text entered by the user can be their diary, a transcribed
  conversation, or a script from a show that their current emotional state relates
  to. Using this text, our system then learns about the users’ current emotional
  state and uses that information to predict desired musical themes and songs to
  the user. The purpose of the user study is to verify the efficiency of this
  software system, where the participants are asked to provide a short body of
  text, and following the review of the recommended song list, participants are
  invited to provide feedback on the outcome they received.

  \textbf{Data usage:}

  This research is anonymous, the IP addresses and other personal information
  that could be associated with the participants will be deleted before analyzing
  the data. The data will be kept securely on Amazon Web Services for four years
  for possible further study, and they will be deleted after four years. The
  information required for the registration is the username and password. If you
  wish to receive a copy of the results of the study, please answer “yes” in the
  google survey form under the result disclosure section. A copy of the report of
  the study in the form of the senior thesis project document will be sent to you
  upon the completion of the analysis.

  \textbf{Exiting the study:}

  The potential risks of this study is that the participants may spend a long
  time writing texts. They  may also have to think about sad memories. If you
  feel uncomfortable in any of the sections during the process, you can click the
  quit button. If you decide not to stop the participation, you can email the
  principal investigator of the project at \emph{liux2@allegheny.edu} to request
  immediate deletion of any data you have created.

  \textbf{Compensation:}

  All participants will have the opportunity to join a drawing for a chance to
  win a \$20 Amazon gift card. Please answer “yes” in the google survey form under
  the lottery section if you want your name to be entered into the drawing. All
  participants will receive a notification of the results of the drawing.

  \textbf{Contact information:}

  If you have any questions, please contact us.

  \textbf{Xingbang Liu:}

  \textbf{Phone:} (814)-795-0122

  \textbf{E-mail:} liux2@allegheny.edu

  \textbf{Professor Janyl Jumadinova:}

  \textbf{E-mail:} jjumadinova@allegheny.edu

  If you consent to the participation in this study, please continue with the
  next step.
\end{displayquote}

After participants log in, they will be taken to the “Text Entering” page.

\subparagraph{Text Entering Page}

After participants enter their text according to the instruction, they will be
directed to the “Result” page containing their recommended playlist, below which
information about the survey will be included.

\subparagraph{Result Page}

After listening to the songs recommended, users can proceed to the servey.

\begin{displayquote}
  \textbf{Survey 1:} Please click on this link for the survey to provide feedback
  on your recommended playlist.

  \textbf{Survey 2:} Please click on this link to indicate your interest in receiving
  the results of this study and your interest in entering the drawing.
\end{displayquote}

\section{Evaluation}

\section{Threats to Validity}
